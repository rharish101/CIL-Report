We use the Albumentations~\cite{albumentations} library for all of our augmentations.


% -------- Class ----------------
\subsection{Augmentations for U-Net inputs}\label{appendix:augmentations-class}
Vanilla U-Net \cite{unet} uses elastic deformations as well as random crops as the augmentation set.
Our final sequence of augmentations is listed in Table~\ref{tab:augmentations-class}.
Each individual augmentation in the sequence is applied with the probability of $0.5$.

\begin{table*}
\centering
    \caption{
        Augmentations are applied sequentially in the following order with a probability of 0.5.
    }\label{tab:augmentations-class}
    \begin{tabular}{c l l}
        \toprule
        Class & Augmentation & Hyper-parameters \\
        \midrule
        \multirow{4}{*}{Rotation and Crop} & Random crop & crop size $= 256 \times 256$ \\
        & Random $90^{\circ}$ rotation & \\
        & Horizontal flip & \\
        & Vertical flip & \\
        \midrule
        \multirow{6}{*}{Color transformation} & \multirow[t]{2}{*}{Random brightness and contrast} & brightness $\sim U[-0.2, 0.2]$ \\
        & & contrast limit $\sim U[-0.2, 0.2]$ \\
        & \multirow[t]{2}{*}{Color jitter} & brightness, contrast $= 0.2$\\
        &  & hue, saturation $= 0.2$\\
        & \multirow[t]{2}{*}{Gaussian blur} & kernel size $k \sim \mathcal{U}\{3, 7\}$\\
        &  & standard deviation $\sigma = 0.3*((k-1)*0.5 - 1) + 0.8$\\
        \midrule
        \multirow{7}{*}{Shape deformation$^*$} & \multirow[t]{3}{*}{Elastic deformation} & $\alpha=1$ \\
        & & $\sigma=50$ \\
        & & $\alpha_{affine} \sim U[-50,50]$ \\
        & \multirow[t]{2}{*}{Grid distortion} & number of steps $=3$\\
        & & distortion $\sim U[-0.3, 0.3]$  \\
        & \multirow[t]{2}{*}{Optical distortion} & shift limit $\sim U[-0.05, 0.05]$\\
        & & Distortion limit: $\sim U[-0.05, 0.05]$  \\
        \bottomrule
        \multicolumn{3}{l}{\small * One of the three deformation transformations is randomly picked each time.}
    \end{tabular}
\end{table*}

% -------- Shape ----------------
\subsection{Augmentations for Contrastive Learning}\label{appendix:augmentations-shape}

When interpreting images in the frequency domain, shapes are generally low frequency components, while textures are generally high frequency components.
Since we need augmentations that preserve shape but change the texture, they need to either reduce high frequency components or distort them (e.g., by adding noise).
These are listed in Table~\ref{tab:augmentations-shape}.

\begin{table*}
    \caption{%
        Augmentations for contrastive learning.
        Each augmentation is applied in the following order with a probability of 0.5.
        The input image is assumed to be in $[0, 255]$.
    }\label{tab:augmentations-shape}
    \begin{tabular}{l l}
        \toprule
        Augmentation & Hyper-parameters \\
        \midrule
        \multirow[t]{2}{*}{Gaussian blur} & kernel size $k \sim \mathcal{U}\{3, 7\}$ \\
        & standard deviation $\sigma = 0.3*((k-1)*0.5 - 1) + 0.8$ \\
        Nearest-neighbour downscaling + upscaling & scale $\sim U[0.5, 0.75]$ \\
        JPEG compression & quality $\sim U[20, 50]$ \\
        \multirow[t]{2}{*}{Gaussian noise} & mean = 0 \\
        & variance $\in (10, 50)$ \\
        \multirow[t]{2}{*}{ISO noise} (using Poisson noise) & hue change angle $\sim U[3.6, 18]$ degrees \\
        & intensity $\sim U[0.1, 0.5]$ \\
        \bottomrule
    \end{tabular}
\end{table*}