Our solution includes different novelties at all steps of the machine learning pipeline.
First, the data is cleaned, then we apply a fine-tuned set of augmentations.
The training of our baseline is enriched with contrastive learning, and the outputs go through a post-processing pipeline consisting of Graph Cuts and SPP.

Two of our main novelties are SPP and contrastive learning for shape bias.
SPP makes clever use of the inductive biases of the road segmentation task.
On the other hand, the contrastive learning approach for inducing shape bias is a general approach that can be applied to any CNN-based model to reduce texture bias.

Our approach highlights the strength of both neural and non-neural approaches.
An avenue for future improvements is more fluently linking the different steps into a jointly learned pipeline to enable signal propagation between the final output metrics and the parameters of our core CNN.
Another avenue is to investigate how the proposed novelties can be applied to tasks other than road segmentation.