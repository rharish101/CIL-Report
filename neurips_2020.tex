\documentclass{article}

% if you need to pass options to natbib, use, e.g.:
%     \PassOptionsToPackage{numbers, compress}{natbib}
% before loading neurips_2020

% ready for submission
% \usepackage{neurips_2020}

% to compile a preprint version, e.g., for submission to arXiv, add add the
% [preprint] option:
%     \usepackage[preprint]{neurips_2020}

% to compile a camera-ready version, add the [final] option, e.g.:
%     \usepackage[final]{neurips_2020}

% to avoid loading the natbib package, add option nonatbib:
     \usepackage[final, nonatbib]{neurips_2020}

\usepackage[
backend=biber,
style=numeric,
sorting=none
]{biblatex}
\addbibresource{refs.bib}


\usepackage[utf8]{inputenc} % allow utf-8 input
\usepackage[T1]{fontenc}    % use 8-bit T1 fonts
\usepackage{hyperref}       % hyperlinks
\usepackage{url}            % simple URL typesetting
\usepackage{booktabs}       % professional-quality tables
\usepackage{amsfonts}       % blackboard math symbols
\usepackage{nicefrac}       % compact symbols for 1/2, etc.
\usepackage{microtype}      % microtypography
\usepackage[dvipsnames]{xcolor}
\usepackage{graphicx}
\usepackage{amsmath}
\usepackage{mathtools}
\usepackage{lipsum}
\usepackage{wrapfig}
\usepackage{xcolor}
\usepackage{subcaption}
\DeclareMathOperator*{\argmax}{arg\,max}
\DeclareMathOperator*{\argmin}{arg\,min}


\title{Road Segmentation - SHAJ report}
%\title{Natural Language Retrieval For Language Grounding in Reinforcement Learning}
%\title{Peruse The Loathsome Enchidirion}
%\title{Know Where To Look}
% The \author macro works with any number of authors. There are two commands
% used to separate the names and addresses of multiple authors: \And and \AND.
%
% Using \And between authors leaves it to LaTeX to determine where to break the
% lines. Using \AND forces a line break at that point. So, if LaTeX puts 3 of 4
% authors names on the first line, and the last on the second line, try using
% \AND instead of \And before the third author name.

\author{
  Shabnam Ghasemirad\\
  \texttt{sghasemirad} \\
  % Affiliation \\
  % Address \\
  % \texttt{email} \\
   \And
  Harish Rajagopal \\
  \texttt{hrajagopal} \\
  % Affiliation \\
  % Address \\
  % \texttt{email} \\
   \And
  Ali Gorji \\
  \texttt{agorji} \\
  % Affiliation \\
  % Address \\
  % \texttt{email} \\
   \And
  Johannes Dollinger \\
  \texttt{jdollinger} \\
  % Affiliation \\
  % Address \\
  % \texttt{email} \\
}
\begin{document}

\maketitle

\section{Introduction} \label{intro}
The saying goes: "All roads lead to Rome." Were that the case, then the task of road segmentation would be pretty easy. But alas, it is not. Road segmentation concerns itself with automatically detecting roads and non-roads in bird's-eye view images, thus falling into the field of semantic segmentation in computer vision. Exemplary applications are the automatic generation of maps for navigation systems, which otherwise would have to be annotated by hand.

Previous models~\cite{bastani2018roadtracer, mattyus2017deeproadmapper, kaiser2017learning} are both very data-hungry and biased for straight road segments. The dataset in this work consists of 100 training images of $400 \times 400$ pixels and 96 test images with $608 \times 608$ pixels. This dataset is relatively small and also very noisy. To tackle the noise issue, we manually cleaned the data.

Our approach to the task combines a U-Net~\cite{unet} with Contrastive Learning~\cite{chopra2005learning} and a sequence of data augmentations~\cite{lecun1998gradient}.
In the post-processing step, we then apply Graph Cut ~\cite{graphcut} and a novel algorithm called Simple Physarum Polycephalum (Section~\ref{novelties:spp})\footnote{A documented implementation can be found on: \href{https://github.com/rharish101/CIL-Project}{https://github.com/rharish101/CIL-Project}}.
Section~\ref{novelties:scrapped} discusses multiple novelties that we attempted but decided to scrap due to no significant improvements over the baseline.

\section{Related Work}\label{section:literature_review}
LIT REV

\section{Baseline}\label{section:baselines}
UNET baseline.
Also mention of GAN

\section{Novelties}

    \subsection{Pre-Processing Deformations}

    \subsection{Different Losses}
    soft dice loss
    patch-wise loss
    \subsection{Post-Processing}

        \subsubsection{GNNs}

        \subsubsection{SLIME}

        \subsubsection{CRF}

        \subsubsection{Graph Cuts}

    \subsection{UNET bottleneck}
        In the U-Net architecture, the downsampling at each depth is done using max-pooling.
        However, if the image size is too large, the layers might miss the global information at the lowest depth.
        Hence, we experiment by using global average pooling at the lowest depth.
        This enforces a bottleneck that captures a single scalar for the entire image in each channel.
        This is done to capture global information from the entire image irrespective of the image size.

    \subsection{GAN}

    \subsection{Contrastive Learning in Bottleneck}
    
    \subsection{Learned ensemblers}

\section{Experiments}\label{section:experiments}
\section{Result}\label{section:results}

\section{Discussion}\label{section:model}
DISCUSSION

\newpage

\small
\printbibliography{}

\end{document}
